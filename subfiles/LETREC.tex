\documentclass[../codeprint.tex]{subfiles}
\begin{document}
\subsection{Plain LETREC}
\label{lang:LETREC}
\lstinputlisting[caption=The entire LETREC interpreter]{chapter3/LETREC/LETREC.rkt}
\lstinputlisting[firstline=1, lastline=33, caption=Syntax file for LETREC]{chapter3/LETREC/syntax.rkt}

\subsection{DYNAMIC SCOPING LETREC}
\label{scoping}
\lstinputlisting[caption=The entire LETREC interpreter]{chapter3/LETREC/LETREC.rkt}
\lstinputlisting[linerange={16-21,105-110,135-145}, caption=Dynamic scopign LETREC interpreter.]{examsolution/2014-01-25-q1-SCOPING/LETREC-dynamic.rkt}
\lstinputlisting[caption=Example of static vs. dynamic scoping.]{examsolution/2014-01-25-q1-SCOPING/scoping.rkt}

\subsection{LETREC-CONT: continuation-passing LETREC}
\label{continuation}
This implementation is identical to LETREC (see \autoref{lang:LETREC}) except that the control-flow is explicit by making use of continuations. A trampolining interpreter is not shown but can be found at \cite[p.~155-166]{Friedman:2008:EPL:1378240}.
\lstinputlisting[linerange={104-139,184-275}, caption=Continuation passing interpreter for LETREC.]{chapter5/LETREC-CONT/LETREC-CONT.rkt}

\subsection{LETREC-EXCEP: LETREC with exceptions}
This language is LETREC-CONT (see \autoref{continuation}) extended with lists and exceptions.
\lstinputlisting[linerange={1-146,178-340}, caption=Interpreter for LETREC that supports lists and exceptions.]{chapter5/EXCEPTIONS/EXCEPTIONS.rkt}
\lstinputlisting[caption=Syntax for LETREC-EXCEP.]{chapter5/EXCEPTIONS/syntax.rkt}
\end{document}
